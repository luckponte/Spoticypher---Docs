\chapter{Introdução}
\label{c.introducao}

Música é uma parte integral da cultura humana de uma maneira quase universal, e diversos estudos até mesmo apontam uma forte correlação com a psique e as emoções ~\cite{krumhansl02}, tanto como influenciadora de comportamento e sentimentos como influenciada pelos mesmos em sua construção e consumo. Desse modo, é fácil deduzir que fatores externos que também afetam essas áreas do nosso psicológico indiretamente afetam o nosso comportamento como ouvintes e produtores musicais. Desse modo, podemos extender essa conexão comportamental e propor que tendências de composição e popularidade de peças podem ser usadas para medir e entender esses fenômenos econômicos e sociais.

Com o avanço da tecnologia da informação e comunicação surgiu não só a nescessidade de processar dados de maneira global e encontrar métodos mais inteligentes e sofisticados de não apenas processá-los mas também compreender seus paralelos como também veio a demanda e a popularização de serviços \textit{online} para suprir por meio de um mercado virtual ações cotidianas, especialmente na esfera do entretenimento, entre elas assistir vídeos, conversar ou manter contato com pessoas remotas e, naturalmente, ouvir música ~\cite{aljanaki15}.

A esse grande conjunto de dados imprecisos, e por extensão ferramentas, técnicas de processamento e manipulação, e aplicações associadas em seu estudo, denominamos \textit{big data} ~\cite{singh15}. Até avanços maiores nas áreas de comunicação e inteligência pouco se desenvolvia sobre o assunto, dado que a estrutura comum de armazenamento de dados (como bancos de dados) era suficiente para a maior parte das aplicações e que, sem o uso de técnicas de aprendizagem e \textit{hardware} mais robusto, o processamento de tal quantidade de dados seria inviável. Entretanto, a maior integração entre computadores, serviços e aplicações, com a  maior democratização da \textit{internet} e o adventos dos dispositivos \textit{mobile}, bem como evoluções nos campos teórico e prático de IA, viu-se a oportunidade de extender esse campo, o que resultou em não somente soluções práticas para problemas do mundo real, como também em métodos de processamento e mineração de dados também aplicáveis a nível funcional para conjuntos menores de dados.

Como os serviços mencionados acima e eventos de ordem econômico-social geram um grande volume de dados ~\cite{aljanaki15} que pode ser acessado para análise, pode-se usar técnicas de análise de dados provenientes do \textit{big data} para encontrar as relações propostas no primeiro parágrafo entre comportamento de consumo e produção musical e eventos que afetam a sociedade em maior escala. Esse trabalho assim se propõe a utilizar algoritmos e ferramentas de \textit{big data} e \textit{data mining} para montar um \textit{software} que processe e parseie esses dados a fim de determinar se pode-se concluir uma relação definida entre eventos globais e locais e tendências no mundo da música.

