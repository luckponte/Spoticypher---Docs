% --------------------------------------------------------
% DEFINIÇÕES DO DOCUMENTO
% --------------------------------------------------------

\documentclass[
	% -- opções da classe memoir --
	12pt,				% tamanho da fonte
	openright,			% capítulos começam em pág ímpar (insere página vazia caso preciso)
	oneside,			% para impressão em verso e anverso. Oposto a oneside
	a4paper,			% tamanho do papel. 
	% -- opções da classe abntex2 --
	%chapter=TITLE,		% títulos de capítulos convertidos em letras maiúsculas
	%section=TITLE,		% títulos de seções convertidos em letras maiúsculas
	%subsection=TITLE,	% títulos de subseções convertidos em letras maiúsculas
	%subsubsection=TITLE,% títulos de subsubseções convertidos em letras maiúsculas
	% -- opções do pacote babel --
	english,			% idioma adicional para hifenização
	french,				% idioma adicional para hifenização
	spanish,			% idioma adicional para hifenização
	brazil,				% o último idioma é o principal do documento
	]{abntex2}


% --------------------------------------------------------
% PACOTES
% --------------------------------------------------------

\usepackage{cmap}				% Mapear caracteres especiais no PDF
\usepackage{lmodern}			% Usa a fonte Latin Modern			
\usepackage[T1]{fontenc}		% Selecao de codigos de fonte.
\usepackage[utf8]{inputenc}		% Codificacao do documento (conversão automática dos acentos)
\usepackage{lastpage}			% Usado pela Ficha catalográfica
\usepackage{indentfirst}		% Indenta o primeiro parágrafo de cada seção.
\usepackage{color}				% Controle das cores
\usepackage{graphicx}			% Inclusão de gráficos
\usepackage{lipsum}				% para geração de dummy text
\usepackage[brazilian,hyperpageref]{backref}	 % Paginas com as citações na bibl
\usepackage[alf,abnt-etal-list=0]{abntex2cite}	% Citações padrão ABNT
\usepackage[br]{nicealgo}       % Pacote para criação de algoritmos
\usepackage{customizacoes}      % Pacote de customizações do abntex2
\usepackage{listings}
\usepackage[normalem]{ulem} % Strikethrough package


% --------------------------------------------------------
% CONFIGURAÇÕES DE PACOTES
% --------------------------------------------------------

% Configurações do pacote listing
\renewcommand{\lstlistingname}{Código} %Mudança no caption do listing para Código
\renewcommand{\lstlistlistingname}{Lista de códigos} %Mudança no caption da lista de listings.

% Configurações do pacote backref
\renewcommand{\familydefault}{\sfdefault}
% Usado sem a opção hyperpageref de backref
\renewcommand{\backrefpagesname}{Citado na(s) página(s):~}
% Texto padrão antes do número das páginas
\renewcommand{\backref}{}
% Define os textos da citação
\renewcommand*{\backrefalt}[4]{
	\ifcase #1 %
		Nenhuma citação no texto.%
	\or
		Citado na página #2.%
	\else
		Citado #1 vezes nas páginas #2.%
	\fi}%


% --------------------------------------------------------
% INFORMAÇÕES DE DADOS PARA CAPA E FOLHA DE ROSTO
% --------------------------------------------------------

\titulo{Título do Texto}
\autor{Lucas Ponte Correia}
\local{Bauru}
\data{2018}
\orientador{Prof. Dr. Clayton Reginaldo Pereira}
\instituicao{%
  Universidade Estadual Paulista ``Júlio de Mesquita Filho''
  \par
  Faculdade de Ciências
  \par
  Ciência da Computação}
\tipotrabalho{Trabalho de Conclusão de Curso}
\preambulo{Trabalho de Conclusão de Curso do Curso de Ciência da Computação da Universidade Estadual Paulista ``Júlio de Mesquita Filho'', Faculdade de Ciências, Campus Bauru.}


% --------------------------------------------------------
% CONFIGURAÇÕES PARA O PDF FINAL
% --------------------------------------------------------

% alterando o aspecto da cor azul
\definecolor{blue}{RGB}{41,5,195}

% informações do PDF
\makeatletter
\hypersetup{
     	%pagebackref=true,
		pdftitle={\@title}, 
		pdfauthor={\@author},
    	pdfsubject={\imprimirpreambulo},
	    pdfcreator={LaTeX with abnTeX2},
		pdfkeywords={abnt}{latex}{abntex}{abntex2}{trabalho acadêmico}, 
		colorlinks=true,       		% false: boxed links; true: colored links
    	linkcolor=blue,          	% color of internal links
    	citecolor=blue,        		% color of links to bibliography
    	filecolor=magenta,      		% color of file links
		urlcolor=blue,
		bookmarksdepth=4
}
\makeatother


% --------------------------------------------------------
% ESPAÇAMENTOS ENTRE LINHAS E PARÁGRAFOS
% --------------------------------------------------------

% O tamanho do parágrafo é dado por:
\setlength{\parindent}{1.3cm}

% Controle do espaçamento entre um parágrafo e outro:
\setlength{\parskip}{0.2cm}


% --------------------------------------------------------
% COMPILANDO O ÍNDICE
% --------------------------------------------------------

\makeindex


% --------------------------------------------------------
% INÍCIO DO DOCUMENTO
% --------------------------------------------------------

\begin{document}

% Seleciona o idioma do documento (conforme pacotes do babel)
\selectlanguage{brazil}

% Retira espaço extra obsoleto entre as frases.
\frenchspacing 


% --------------------------------------------------------
% ELEMENTOS PRÉ-TEXTUAIS
% --------------------------------------------------------

% Capa
\imprimircapa

% Folha de rosto
% (o * indica que haverá a ficha bibliográfica)
\imprimirfolhaderosto*

% Inserir a ficha bibliografica
\begin{fichacatalografica}
	\sffamily
	\vspace*{\fill}					% Posição vertical
	\begin{center}					% Minipage Centralizado
	\fbox{\begin{minipage}[c][8cm]{15.5cm}		% Largura
	\small
	\imprimirautor
	\hspace{0.5cm} \imprimirtitulo  / \imprimirautor. --
	\imprimirlocal, \imprimirdata-
	\hspace{0.5cm} \pageref{LastPage} p. : il. (algumas color.) ; 30 cm.\\
	\hspace{0.5cm} \imprimirorientadorRotulo~\imprimirorientador\\
	\hspace{0.5cm}
	\parbox[t]{\textwidth}{\imprimirtipotrabalho~--~\imprimirinstituicao,
	\imprimirdata.}\\
	\hspace{0.5cm}
		1. Tags
		2. Para
		3. A
		4. Ficha
		5. Catalográfica	
	\end{minipage}}
	\end{center}
\end{fichacatalografica}

% Inserir folha de aprovação
\begin{folhadeaprovacao}
  \begin{center}
    {\ABNTEXchapterfont\large\imprimirautor}
    \vspace*{\fill}\vspace*{\fill}
    \begin{center}
      \ABNTEXchapterfont\bfseries\Large\imprimirtitulo
    \end{center}
    \vspace*{\fill}
    \hspace{.45\textwidth}
    \begin{minipage}{.5\textwidth}
        \imprimirpreambulo
    \end{minipage}%
    \vspace*{\fill}
  \end{center}
  \center Banca Examinadora
  \assinatura{\textbf{\imprimirorientador} \\ Orientador} 
  \assinatura{\textbf{Professor} \\ Convidado 1}
  \assinatura{\textbf{Professor} \\ Convidado 2}
  \begin{center}
    \vspace*{0.5cm}
    \par
    {Bauru, \_\_\_\_\_ de \_\_\_\_\_\_\_\_\_\_\_ de \_\_\_\_.}
    \vspace*{1cm}
  \end{center}
\end{folhadeaprovacao}

% Dedicatória
\begin{dedicatoria}
   \vspace*{\fill}
   \centering
   \noindent
   \textit{Espaço destinado à dedicátoria do texto.} \vspace*{\fill}
\end{dedicatoria}

% Agradecimentos
\begin{agradecimentos}
Espaço destinado aos agradecimentos.
\end{agradecimentos}

% Epígrafe
\begin{epigrafe}
    \vspace*{\fill}
	\begin{flushright}
		\textit{Espaço destinado à epígrafe.}
	\end{flushright}
\end{epigrafe}


% --------------------------------------------------------
% RESUMOS
% --------------------------------------------------------

% resumo em português
\setlength{\absparsep}{18pt} % ajusta o espaçamento dos parágrafos do resumo
\begin{resumo}
Espaço destinado à escrita do resumo.
\textbf{Palavras-chave:} Palavras-chave de seu resumo.
\end{resumo}

% % resumo em inglês
\begin{resumo}[Abstract]
 \begin{otherlanguage*}{english}
Abstract area.
\textbf{Keywords:} Abstract keywords.
 \end{otherlanguage*}
\end{resumo}  


% --------------------------------------------------------
% LISTA DE ILUSTRAÇÕES
% --------------------------------------------------------

% inserir lista de ilustrações
\pdfbookmark[0]{\listfigurename}{lof}
\listoffigures*
\cleardoublepage


% --------------------------------------------------------
% LISTA DE TABELAS
% --------------------------------------------------------

% inserir lista de tabelas
\pdfbookmark[0]{\listtablename}{lot}
\listoftables*
\cleardoublepage


% --------------------------------------------------------
% LISTA DE CÓDIGOS
% --------------------------------------------------------
 
\counterwithout{lstlisting}{section}
\pdfbookmark[0]{\lstlistlistingname}{lof}
\lstlistoflistings
\cleardoublepage


% --------------------------------------------------------
% LISTA DE ABREVIATURAS E SIGLAS
% --------------------------------------------------------

% inserir lista de abreviaturas e siglas
\begin{siglas}
 \item[Fig.] Area of the $i^{th}$ component
 \item[456] Isto é um número
 \item[123] Isto é outro número
 \item[lauro cesar] este é o meu nome
\end{siglas}

% --------------------------------------------------------
% LISTA DE SÍMBOLOS
% --------------------------------------------------------

% inserir lista de símbolos
\begin{simbolos}
  \item[$ \Gamma $] Letra grega Gama
  \item[$ \Lambda $] Lambda
  \item[$ \zeta $] Letra grega minúscula zeta
  \item[$ \in $] Pertence
\end{simbolos}


% --------------------------------------------------------
% SUMÁRIO
% --------------------------------------------------------

% inserir o sumario
\pdfbookmark[0]{\contentsname}{toc}
\tableofcontents*
\cleardoublepage


% --------------------------------------------------------
% ELEMENTOS TEXTUAIS
% --------------------------------------------------------

\pagestyle{simple}

% Arquivos .tex do texto, podendo ser escritos em um único arquivo ou divididos da forma desejada
\chapter{Introdução}
\label{c.introducao}

\sout{Para iniciar a produção em .tex é necessário instalar os pacotes básicos da linguagem e seus compiladores. O MiKTeX é um pacote básico para o Windows (miktex.org/download) e o MacTeX um pacote básico para o Mac (tug.org/mactex) que contém o mínimo necessário de TeX/LaTeX para rodar. Ele já vem com os compiladores nativos da linguagem e uma IDE (TeXworks, para edição do texto) que possui o compilador integrado.}

\sout{Normalmente é utilizado o modo pdfLaTeX + MakeIndex + BibTeX para compilar um arquivo .tex. Existem outros formatos de compiladores, mas essa opção é capaz de gerar um .pdf automático após a compilação e ainda por cima adicionar as funcionalidades do BibTeX (recursos para criação e montagem automática de fontes bibliográficas).}

\sout{Além disso, também é necessária a instalação do pacote abnTeX2. Esse tutorial https://github.com/abntex/abntex2/wiki/Instalacao provém o passo a passo de como instalar cada componente do TeX, em qualquer sistema operacional (Linux, Mac OS e Windows). Caso esteja utilizando o MiKTeX, ele é capaz de efetuar o download do pacote automaticamente, apenas instale-o, abra o projeto.tex e compile-o, ele irá requisitar a autorização para baixar automaticamente os pacotes que faltam para efetuar a compilação.} \textcolor{red}{NÃO MAIS!}

Com tudo em mãos e o compilador funcionando, é hora de abrir o modelo (projeto.tex) e começar a escrever o texto. É possível perceber no código a estrutura do arquivo e os campos possíveis de edição. Ao escrever o texto, ele é escrito normalmente, sendo que existem diversos comandos para estilizá-lo, criar tabelas, figuras, dentre outros. A seguir abordaremos os principais comandos e funções que podem ser utilizadas em um projeto básico de TCC. Para outras funções e pacotes, procure no Google, a comunidade é ativa e provavelmente já deve ter feito o que é de sua necessidade.

O arquivo projeto.tex contém os pacotes e comandos básicos que definem a estrutura desse texto já no formato requisitado pela ABNT. Dentro dele é possível ver que estamos importando outros dois arquivos .tex (introducao e conclusao), ou seja, esses arquivos estão sendo basicamente concatenados com o comando ``input''. A divisão não é necessária, mas pode ser que auxilie na escrita do texto  ao deixar as coisas mais separadas e organizadas, não sendo um único arquivo cheio de linhas e linhas de código. 

\section{Modificadores de Texto}
\label{s.modificador}

Os modificadores de texto mais simples utilizados são o negrito (``textbf'') \textbf{texto em negrito} e o itálico (``emph'') \emph{texto em itálico}.

\section{Seções}
\label{s.citacoes}

Seções podem ser criadas a partir do comando ``section'' e hierarquizadas abaixo do capítulo principal. É possível referenciá-las, por exemplo, Seção~\ref{s.citacoes} corresponde a seção atual em que estamos. Já se quisermos referenciar alguma outra coisa, é só utilizarmos o comando ``ref'' presente no código desse texto, por exemplo, Capítulo~\ref{c.introducao}.

\subsection{Subseções}
\label{ss.subsecao}

Subseções também podem ser criadas com o comando ``subsection'' e referenciadas~\ref{ss.subsecao}.

\subsubsection{Sub-subseções}
\label{sss.subsubsecao}

Também há mais um nível que pode ser criado com o comando ``subsubsection''.

\section{Alíneas}
\label{s.alineas}

\begin{alineas}

\item As alineas devem ser criadas desse modo, com o comando begin\{alineas\}. Isso é necessário para que estejam no formato definido pelo pacote abnTeX2 e, consequentemente, no formato definido pela ABNT.

\item Cada item da alínea pode ser invocado com um comando item.

\item O fim de cada alínea é determinado por end\{alineas\}.

\end{alineas}

\section{Tabelas}
\label{s.tabelas}

As tabelas também podem ser referenciadas como se fossem seções ou figuras, por exemplo, esta é a Tabela~\ref{t.transacao_mercado}.

\begin{table}[h]
\centering
\begin{tabular}{c|c}
\hline
\textbf{\small TID} & \textbf{\small Conjunto de Itens}\\\hline \hline
{\small 1} & {\small \{Pão, Leite\}}\\\hline
{\small 2} & {\small \{Pão, Fralda, Cerveja, Ovos\}}\\\hline
{\small 3} & {\small \{Leite, Fralda, Cerveja, Coca-Cola\}}\\\hline
{\small 4} & {\small \{Pão, Leite, Fralda, Cerveja\}}\\\hline
{\small 5} & {\small \{Pão, Leite, Fralda, Coca-Cola\}}\\\hline
\end{tabular}
\caption{Exemplo de transações de mercado.}
\label{t.transacao_mercado}
\end{table}

Quando uma tabela é criada com begin\{table\}, ela é automaticamente adicionada à Lista de Tabelas.

\section{Algoritmos}
\label{s.algoritmos}

O pacote nicealgo incluído nos arquivos desse projeto é responsável por disponibilizar comandos extras, não inerentes ao básico TeX, para a criação de algoritmos. Um exemplo do Algoritmo~\ref{a.algoritmo} é escrito a seguir. Eles também pode ser referenciados como se fossem tabelas ou figuras.

\begin{nicealgo}{a.algoritmo}
\naTITLE{Algoritmo AIS}
\naPREAMBLE
\naINPUT{Conjunto Frequente L = 0 e Grupo de Fronteira F = 0.}
\naBODY
\naBEGIN{\textbf{Enquanto} $F \neq 0$, \textbf{faça}}
\na{\textbf{Seja} conjunto candidato $C = 0$;}
\naBEGIN{\textbf{Para cada} tuplas $t$ da base de dados, \textbf{faça}}
\naBEGIN{\textbf{Para cada} conjuntos de itens $f$ em $F$, \textbf{faça}}
\naBEGIN{\textbf{Se} $t$ contém $f$, \textbf{então}}
\naEND{\textbf{Seja} $C_f =$ conjuntos de itens candidatos extensões de $f$ e contidos em $t$;}
\naBEGIN{\textbf{Para cada} conjunto de itens $c_f$ em $C_f$, \textbf{faça}}
\naBEGIN{\textbf{Se} $c_f \in C$, \textbf{então}}
\naEND{$c_f$.contagem $= c_f$.contagem$ + 1$;}
\naBEGIN{\textbf{Se não}}
\na{$c_f$.contagem $= 0$;}
\naEND{$C = C + c_f$;}
\naEND{}
\naEND{}
\naEND{}
\na{\textbf{Seja} F = 0;}
\naBEGIN{\textbf{Para cada} conjunto de itens $c$ em $C$, \textbf{faça}}
\naBEGIN{\textbf{Se} $contagem(c)/tamanho\_db > minsupport$, \textbf{então}}
\naEND{$L = L + c$;}
\naBEGIN{\textbf{Se} $c$ deve ser usado como a próxima fronteira, \textbf{então}}
\naEND{$F = F + c$;}
\naEND{}
\naEND{}
\end{nicealgo}

\section{Códigos}
\label{s.codigos}

Códigos podem ser criados a partir do comando begin\{lstlisting\} e end\{lstlisting\}. É possível passar parâmetros para essa função, como por exemplo, a linguage do código e a legenda dele. Por exemplo: \char`\\begin\{lstlisting\}[language=Python, caption=Exemplo de código em Python]

\begin{lstlisting}[language=Python, caption=Exemplo de código em Python]
import numpy as np
 
def incmatrix(genl1,genl2):
    m = len(genl1)
    n = len(genl2)
    M = None #to become the incidence matrix
    VT = np.zeros((n*m,1), int)  #dummy variable
 
    #compute the bitwise xor matrix
    M1 = bitxormatrix(genl1)
    M2 = np.triu(bitxormatrix(genl2),1) 
 
    for i in range(m-1):
        for j in range(i+1, m):
            [r,c] = np.where(M2 == M1[i,j])
            for k in range(len(r)):
                VT[(i)*n + r[k]] = 1;
                VT[(i)*n + c[k]] = 1;
                VT[(j)*n + r[k]] = 1;
                VT[(j)*n + c[k]] = 1;
 
                if M is None:
                    M = np.copy(VT)
                else:
                    M = np.concatenate((M, VT), 1)
 
                VT = np.zeros((n*m,1), int)
 
    return M
\end{lstlisting}

\section{Figuras}
\label{s.figuras}

Abaixo podemos identificar a criação e referência da Figura~\ref{f.disposicao_mercado}. Atente-se ao código para perceber um possível redimensionamento com a função scale e o caminho de onde a figura deve ser retirada.

Quando uma figura é criada com begin\{figure\}, ela é automaticamente adicionada à Lista de Ilustrações.

\begin{figure}[h]
\caption{\small Exemplo do ambiente TeXworks.}
\centering
\includegraphics[scale=0.50]{figs/tex_exemplo.png}
\label{f.disposicao_mercado}
\legend{\small Fonte: Elaborada pelo autor.}
\end{figure}

\section{Equações}
\label{s.equacoes}

O TeX também é muito famoso pela forma em que consegue tratar funções e símbolos matemáticos. A partir da utiização de dois cifrões (\$codigo matemático\$) é possível identificar ao compilador que a escrita a seguir são símbolos e códigos originários do pacote matemático do TeX. Aqui estamos demonstrado um exemplo $\phi = 1 + x$ dessa utilização.

Também podemos definir equações utilizando os comandos begin\{equation\} e end\{equation\}. Por exemplo:

\begin{equation}
\label{e.energy_rbm}
E(\textbf{v},\textbf{h})=-\sum_{i=1}^ma_iv_i-\sum_{j=1}^nb_jh_j-\sum_{i=1}^m\sum_{j=1}^nv_ih_jw_{ij},
\end{equation}

\begin{equation}
\label{e.probability_configuration}
P(\textbf{v},\textbf{h})=\frac{e^{-E(\textbf{v},\textbf{h})}}{\displaystyle\sum_{\textbf{v},\textbf{h}}e^{-E(\textbf{v},\textbf{h})}},
\end{equation}

\begin{eqnarray}
\label{eq:par}
\hat{\phi}^j & = & \left\{ \begin{array}{ll} \hat{\phi}^j\pm \varphi_j \varrho  & \mbox{{ com probabilidade PAR}} \\
    \hat{\phi}^j & \mbox{{com probabilidade (1-PAR).}}
\end{array}\right.
\end{eqnarray}

Existem diversos sites no Google que contém códigos de símbolos e funções matemáticas de todos os tipos. Exemplo:\\
\begin{center}
\tiny estudijas.lu.lv/pluginfile.php/14809/mod\_page/content/16/instrukcijas/matematika\_moodle/LaTeX\_Symbols.pdf.
\end{center}

\section{Como citar as referências}
\label{ss.referencias}

Aqui está um exemplo de como podemos referenciar as bibliografias utilizadas no trabalho. Elas são guardadas na forma de metadados (tags) no arquivo .bib a qual é importada no projeto principal (projeto.tex).

E podemos citá-las de acordo com os identificadores atribuídos para cada referência, por exemplo,~\cite{stonebraker93} e~\cite{rocha09}.

Após citar um item de referência bibliográfica com o comando ``cite'', ela será automaticamente padronizada e incluída na página de Referências de seu arquivo. Atualmente os maiores sites portadores de artigos, periódicos, dentre outros (IEEE, Springer, etc) já conseguem exportar a publicação desejado no formato BibTeX, sendo facilmente adicionado ao arquivo .bib de seu trabalho.

\chapter{Conclusão}
\label{c.conclusao}

Conforme mencionado, apesar de alguns dos resultados parecerem promissores, não pode-se com segurança confirmar uma correlação entre os dados econômicos e as características das faixas extraídos. Como pontos de consideração sobre como melhorar o desempenho em geral destacam-se, a partir do próprio relatório de erros dos experimentos, a inadequação do modelo, que muito provavelmente advém do uso de um algoritmo sub-ótimo para o problema em questão. Apesar do poder do método de SVM para classificação e da possibilidade de seu uso para regressão (como foi feito), ficou em dúvida se outros algoritmos de regressão, especialmente os especializados em tratar múltipla variáveis, não seriam mais adequados e, por consequente, retornariam resultados mais sólidos. Da mesma forma, apesar do cuidado na manipulação e extração dos dados, não se pode excluir a hipótese de viés nas amostras ou de escolha imprecisa das variáveis, tanto as econômicas quanto as pertinentes às qualidades das músicas. Também pode-se oferecer o argumento que, pesquisando o contexto de outras regiões que não apenas o Brasil poderiam se obter resultados mais conclusivos, já que o número de dados e contextos diferentes minimizariam, em tese, a influência de outros possíveis fatores nos resultados.


% --------------------------------------------------------
% ELEMENTOS PÓS-TEXTUAIS
% --------------------------------------------------------

\postextual


% --------------------------------------------------------
% REFERÊNCIAS BIBLIOGRÁFICAS
% --------------------------------------------------------

\bibliography{referencias}

% --------------------------------------------------------
% GLOSSÁRIO
% --------------------------------------------------------

% Consulte o manual da classe abntex2 para orientações sobre o glossário.
% \glossary

% --------------------------------------------------------
% APÊNDICES
% --------------------------------------------------------

% Inicia os apêndices
\begin{apendicesenv}
% Imprime uma página indicando o início dos apêndices
\partapendices
% Criação do apêndice
\end{apendicesenv}


% --------------------------------------------------------
% ÍNDICE REMISSIVO
% --------------------------------------------------------

\printindex


% --------------------------------------------------------
% FINAL DO DOCUMENTO
% --------------------------------------------------------

\end{document}
