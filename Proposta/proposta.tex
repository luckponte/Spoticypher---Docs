% --------------------------------------------------------
% DEFINIÇÕES DO DOCUMENTO
% --------------------------------------------------------

\documentclass[
	% -- opções da classe memoir --
	12pt,				% tamanho da fonte
	openright,			% capítulos começam em pág ímpar (insere página vazia caso preciso)
	oneside,			% para impressão em verso e anverso. Oposto a oneside
	a4paper,			% tamanho do papel. 
	% -- opções da classe abntex2 --
	%chapter=TITLE,		% títulos de capítulos convertidos em letras maiúsculas
	%section=TITLE,		% títulos de seções convertidos em letras maiúsculas
	%subsection=TITLE,	% títulos de subseções convertidos em letras maiúsculas
	%subsubsection=TITLE,% títulos de subsubseções convertidos em letras maiúsculas
	% -- opções do pacote babel --
	english,			% idioma adicional para hifenização
	french,				% idioma adicional para hifenização
	spanish,			% idioma adicional para hifenização
	brazil,				% o último idioma é o principal do documento
	]{abntex2}


% --------------------------------------------------------
% PACOTES
% --------------------------------------------------------

\usepackage{cmap}				% Mapear caracteres especiais no PDF
\usepackage{lmodern}			% Usa a fonte Latin Modern			
\usepackage[T1]{fontenc}		% Selecao de codigos de fonte.
\usepackage[utf8]{inputenc}		% Codificacao do documento (conversão automática dos acentos)
\usepackage{lastpage}			% Usado pela Ficha catalográfica
\usepackage{indentfirst}		% Indenta o primeiro parágrafo de cada seção.
\usepackage{color}				% Controle das cores
\usepackage{graphicx}			% Inclusão de gráficos
\usepackage{lipsum}				% para geração de dummy text
\usepackage[brazilian,hyperpageref]{backref}	 % Paginas com as citações na bibl
\usepackage[alf,abnt-etal-list=0]{abntex2cite}	% Citações padrão ABNT
\usepackage[br]{nicealgo}       % Pacote para criação de algoritmos
\usepackage{customizacoes}      % Pacote de customizações do abntex2
\usepackage{listings}
\usepackage[normalem]{ulem} % Strikethrough package


% --------------------------------------------------------
% CONFIGURAÇÕES DE PACOTES
% --------------------------------------------------------

% Configurações do pacote listing
\renewcommand{\lstlistingname}{Código} %Mudança no caption do listing para Código
\renewcommand{\lstlistlistingname}{Lista de códigos} %Mudança no caption da lista de listings.

% Configurações do pacote backref
\renewcommand{\familydefault}{\sfdefault}
% Usado sem a opção hyperpageref de backref
\renewcommand{\backrefpagesname}{Citado na(s) página(s):~}
% Texto padrão antes do número das páginas
\renewcommand{\backref}{}
% Define os textos da citação
\renewcommand*{\backrefalt}[4]{
	\ifcase #1 %
		Nenhuma citação no texto.%
	\or
		Citado na página #2.%
	\else
		Citado #1 vezes nas páginas #2.%
	\fi}%


% --------------------------------------------------------
% INFORMAÇÕES DE DADOS PARA CAPA E FOLHA DE ROSTO
% --------------------------------------------------------

\titulo{Análise de dados para estudo da relação entre tendências musicais e fenômenos socio-econômicos}
\autor{Lucas Ponte Correia}
\local{Bauru}
\data{2018}
\orientador{Prof. Dr. Clayton Reginaldo Pereira}
\instituicao{%
  Universidade Estadual Paulista ``Júlio de Mesquita Filho''
  \par
  Faculdade de Ciências
  \par
  Ciência da Computação}
\tipotrabalho{Trabalho de Conclusão de Curso}
\preambulo{Proposta para Trabalho de Conclusão de Curso do Curso de Ciência da Computação da Universidade Estadual Paulista ``Júlio de Mesquita Filho'', Faculdade de Ciências, Campus Bauru.}


% --------------------------------------------------------
% CONFIGURAÇÕES PARA O PDF FINAL
% --------------------------------------------------------

% alterando o aspecto da cor azul
\definecolor{blue}{RGB}{41,5,195}

% informações do PDF
\makeatletter
\hypersetup{
     	%pagebackref=true,
		pdftitle={\@title}, 
		pdfauthor={\@author},
    	pdfsubject={\imprimirpreambulo},
	    pdfcreator={LaTeX with abnTeX2},
		pdfkeywords={abnt}{latex}{abntex}{abntex2}{trabalho acadêmico}, 
		colorlinks=true,       		% false: boxed links; true: colored links
    	linkcolor=blue,          	% color of internal links
    	citecolor=blue,        		% color of links to bibliography
    	filecolor=magenta,      		% color of file links
		urlcolor=blue,
		bookmarksdepth=4
}
\makeatother


% --------------------------------------------------------
% ESPAÇAMENTOS ENTRE LINHAS E PARÁGRAFOS
% --------------------------------------------------------

% O tamanho do parágrafo é dado por:
\setlength{\parindent}{1.3cm}

% Controle do espaçamento entre um parágrafo e outro:
\setlength{\parskip}{0.2cm}


% --------------------------------------------------------
% COMPILANDO O ÍNDICE
% --------------------------------------------------------

\makeindex


% --------------------------------------------------------
% INÍCIO DO DOCUMENTO
% --------------------------------------------------------

\begin{document}

% Seleciona o idioma do documento (conforme pacotes do babel)
\selectlanguage{brazil}

% Retira espaço extra obsoleto entre as frases.
\frenchspacing 


% --------------------------------------------------------
% ELEMENTOS PRÉ-TEXTUAIS
% --------------------------------------------------------

% Capa
\imprimircapa

% Folha de rosto
% (o * indica que haverá a ficha bibliográfica)
\imprimirfolhaderosto*

% Inserir a ficha bibliografica
% \begin{fichacatalografica}
% 	\sffamily
% 	\vspace*{\fill}					% Posição vertical
% 	\begin{center}					% Minipage Centralizado
% 	\fbox{\begin{minipage}[c][8cm]{15.5cm}		% Largura
% 	\small
% 	\imprimirautor
% 	\hspace{0.5cm} \imprimirtitulo  / \imprimirautor. --
% 	\imprimirlocal, \imprimirdata-
% 	\hspace{0.5cm} \pageref{LastPage} p. : il. (algumas color.) ; 30 cm.\\
% 	\hspace{0.5cm} \imprimirorientadorRotulo~\imprimirorientador\\
% 	\hspace{0.5cm}
% 	\parbox[t]{\textwidth}{\imprimirtipotrabalho~--~\imprimirinstituicao,
% 	\imprimirdata.}\\
% 	\hspace{0.5cm}
% 		1. Tags
% 		2. Para
% 		3. A
% 		4. Ficha
% 		5. Catalográfica	
% 	\end{minipage}}
% 	\end{center}
% \end{fichacatalografica}

% Inserir folha de aprovação
% \begin{folhadeaprovacao}
%   \begin{center}
%     {\ABNTEXchapterfont\large\imprimirautor}
%     \vspace*{\fill}\vspace*{\fill}
%     \begin{center}
%       \ABNTEXchapterfont\bfseries\Large\imprimirtitulo
%     \end{center}
%     \vspace*{\fill}
%     \hspace{.45\textwidth}
%     \begin{minipage}{.5\textwidth}
%         \imprimirpreambulo
%     \end{minipage}%
%     \vspace*{\fill}
%   \end{center}
%   \center Banca Examinadora
%   \assinatura{\textbf{\imprimirorientador} \\ Orientador} 
%   \assinatura{\textbf{Professor} \\ Convidado 1}
%   \assinatura{\textbf{Professor} \\ Convidado 2}
%   \begin{center}
%     \vspace*{0.5cm}
%     \par
%     {Bauru, \_\_\_\_\_ de \_\_\_\_\_\_\_\_\_\_\_ de \_\_\_\_.}
%     \vspace*{1cm}
%   \end{center}
% \end{folhadeaprovacao}

% Dedicatória
%\begin{dedicatoria}
   %\vspace*{\fill}
   %\centering
   %\noindent
   %\textit{Espaço destinado à dedicátoria do %texto.} \vspace*{\fill}
%\end{dedicatoria}

% Agradecimentos
%\begin{agradecimentos}
%Espaço destinado aos agradecimentos.
%\end{agradecimentos}

% Epígrafe
%\begin{epigrafe}
  %  \vspace*{\fill}
	%\begin{flushright}
	%	\textit{Espaço destinado à epígrafe.}
	%\end{flushright}
%\end{epigrafe}


% --------------------------------------------------------
% RESUMOS
% --------------------------------------------------------

% resumo em português
%\setlength{\absparsep}{18pt} % ajusta o espaçamento dos parágrafos do resumo
% \begin{resumo}
% Espaço destinado à escrita do resumo.
% \textbf{Palavras-chave:} Palavras-chave de seu resumo.
% \end{resumo}

% % resumo em inglês
% \begin{resumo}[Abstract]
%  \begin{otherlanguage*}{english}
% Abstract area.
% \textbf{Keywords:} Abstract keywords.
%  \end{otherlanguage*}
% \end{resumo}  


% --------------------------------------------------------
% LISTA DE ILUSTRAÇÕES
% --------------------------------------------------------

% inserir lista de ilustrações
%\pdfbookmark[0]{\listfigurename}{lof}
%\listoffigures*
%\cleardoublepage


% --------------------------------------------------------
% LISTA DE TABELAS
% --------------------------------------------------------

% inserir lista de tabelas
%\pdfbookmark[0]{\listtablename}{lot}
%\listoftables*
%\cleardoublepage


% --------------------------------------------------------
% LISTA DE CÓDIGOS
% --------------------------------------------------------
 
%\counterwithout{lstlisting}{section}
%\pdfbookmark[0]{\lstlistlistingname}{lof}
%\lstlistoflistings
%\cleardoublepage


% --------------------------------------------------------
% LISTA DE ABREVIATURAS E SIGLAS
% --------------------------------------------------------

% inserir lista de abreviaturas e siglas
%\begin{siglas}
%  \item[Fig.] Area of the $i^{th}$ component
%  \item[456] Isto é um número
%  \item[123] Isto é outro número
%  \item[lauro cesar] este é o meu nome
%\end{siglas}

% --------------------------------------------------------
% LISTA DE SÍMBOLOS
% --------------------------------------------------------

% inserir lista de símbolos
% \begin{simbolos}
%   \item[$ \Gamma $] Letra grega Gama
%   \item[$ \Lambda $] Lambda
%   \item[$ \zeta $] Letra grega minúscula zeta
%   \item[$ \in $] Pertence
% \end{simbolos}


% --------------------------------------------------------
% SUMÁRIO
% --------------------------------------------------------

% inserir o sumario
\pdfbookmark[0]{\contentsname}{toc}
\tableofcontents*
\cleardoublepage


% --------------------------------------------------------
% ELEMENTOS TEXTUAIS
% --------------------------------------------------------

\pagestyle{simple}

% Arquivos .tex do texto, podendo ser escritos em um único arquivo ou divididos da forma desejada
\chapter{Introdução}
\label{c.introducao}

Música é uma parte integral da cultura humana de uma maneira quase universal, e diversos estudos apontam uma forte correlação com a psique e as emoções ~\cite{krumhansl02}, tanto como influenciadora de comportamento e sentimentos como influenciada pelos mesmos em sua construção e consumo. Desse modo, é fácil deduzir que fatores externos que também afetam essas áreas do nosso psicológico indiretamente afetam o nosso comportamento como ouvintes e produtores musicais. Desse modo, pode-s extender essa conexão comportamental e propor que tendências de composição e popularidade de peças podem ser usadas para medir e entender esses fenômenos econômicos e sociais.

Com o avanço da tecnologia da informação e comunicação, surgiu não só a necessidade de processar dados de maneira global e encontrar métodos mais inteligentes e sofisticados de processá-los para compreender seus paralelos, como também popularizam-se serviços \textit{online} para suprir, por meio de um mercado virtual, ações cotidianas, especialmente na esfera do entretenimento, entre elas: assistir vídeos, conversar ou manter contato com pessoas remotas e, naturalmente, ouvir música ~\cite{aljanaki15}.

Esse grande conjunto de dados imprecisos, e por extensão ferramentas, técnicas de processamento e manipulação, e aplicações associadas em seu estudo, denomina-se \textit{big data} ~\cite{singh15}. Até avanços maiores nas áreas de comunicação e inteligência pouco se desenvolvia sobre o assunto, dado que as estruturas comuns de armazenamento de dados (como bancos de dados) eram suficientes para a maior parte das aplicações e que, sem o uso de técnicas de aprendizagem e \textit{hardware} mais avançado e de processamento mais veloz, o processamento de tal quantidade de dados seria inviável. Entretanto, com a maior integração entre computadores, serviços e aplicações, com a  maior democratização da \textit{internet} e o advento dos dispositivos \textit{mobile}, bem como evoluções nos campos teórico e prático de IA, viu-se a oportunidade de extender esse campo, o que resultou não somente em soluções práticas para problemas do mundo real, como também em métodos de processamento e mineração de dados também aplicáveis a nível funcional para conjuntos menores de dados.

Como produtos e serviços fonográficos baseados em \textit{streaming} (isto é, a transmissão direta de dados pela \textit{internet}) e eventos de ordem econômico-social geram um grande volume de dados ~\cite{aljanaki15} que pode ser acessado para análise, pode-se usar técnicas de análise de dados provenientes do \textit{big data} para encontrar as relações propostas no primeiro parágrafo entre comportamento de consumo e produção musical e eventos que afetam a sociedade em maior escala. Esse trabalho assim se propõe a utilizar algoritmos e ferramentas de \textit{big data} e \textit{data mining} para montar um \textit{software} que processe e analise esses dados a fim de determinar se é possível traçar uma relação definida entre eventos econômicos globais e locais e tendências no mundo da música.

\chapter{Problema}
\label{c.problema}

Conforme discorrido, é conhecido que música gera uma resposta emocional no ouvinte e assim reflete muito sobre seu contexto, de modo que é comumente usada como objeto de estudo para melhor entender a época e o contexto socioeconômico de sua composição. O que se deseja aferir é uma relação mais indireta entre comportamentos relacionados à música, sua produção e seu consumo e o contexto social, político e econômico de um determinado local, de modo a saber se é válido, a partir da análise destes, fazer predições sobre aquele e vice-versa.
    
Com a popularização dos serviços de \textit{streaming} de música (como \textit{Spotify}, \textit{Deezer} e outros) e maior acesso a \textit{data sets} que tratam da situação econômica de uma região em um certo tempo, torna-se viável o uso de técnicas de mineração e análise de dados como ferramentas para buscar correlações entre o ambiente econômico e os tipos de peças que são compostas ou mais populares.
\chapter{Justificativa}
\label{c.justificativa}

Como citado, os processos de análise do contexto econômico e social ainda possuem baixa implementação de processos de automatização, o que consome tempo de pesquisa que poderia ser otimizado com ferramentas de análise de dados que auxiliassem na extração de dados e correlações significativas para o campo de estudo.

A motivação do presente trabalho é a exploração da correlação entre a música em seu contexto popular e prover uma ferramenta que possa auxiliar na extração de significado para campos de pesquisa mais afastados da área de tecnologia, que muitas vezes não se valem de recursos desse tipo.
\section{Objetivos}
\label{c.objetivos}

\subsection{Objetivo Geral}
\label{c.objetivo_g}
Desenvolver um \textit{software} de extração e análise de dados que permita traçar correlações entre o cenário econômico de uma região e a tipografia de suas músicas mais populares em um dado período de tempo.

\subsection{Objetivos Específicos}
\label{c.objetivo_e}
\begin{itemize}
    \item Estudar sobre análise, mineração e redução de dados para \textit{datasets} maiores;
    \item Desenvolver o \textit{software} de redução e análise;
    \item Alimentar o produto com dados provenientes de fontes variadas;
    \item Averiguar a precisão das correlações traçadas para saber se são relevantes o bastante para ser uma ferramenta de apoio.
\end{itemize}

\chapter{Problema}
\label{c.problema}

Conforme discorrido, é conhecido que música gera uma resposta emocional no ouvinte e assim reflete muito sobre seu contexto, de modo que é comumente usada como objeto de estudo para melhor entender a época e o contexto socioeconômico de sua composição. O que se deseja aferir é uma relação mais indireta entre comportamentos relacionados à música, sua produção e seu consumo e o contexto social, político e econômico de um determinado local, de modo a saber se é válido, a partir da análise destes, fazer predições sobre aquele e vice-versa.
    
Com a popularização dos serviços de \textit{streaming} de música (como \textit{Spotify}, \textit{Deezer} e outros) e maior acesso a \textit{data sets} que tratam da situação econômica de uma região em um certo tempo, torna-se viável o uso de técnicas de mineração e análise de dados como ferramentas para buscar correlações entre o ambiente econômico e os tipos de peças que são compostas ou mais populares.
\chapter{Justificativa}
\label{c.justificativa}

Como citado, os processos de análise do contexto econômico e social ainda possuem baixa implementação de processos de automatização, o que consome tempo de pesquisa que poderia ser otimizado com ferramentas de análise de dados que auxiliassem na extração de dados e correlações significativas para o campo de estudo.

A motivação do presente trabalho é a exploração da correlação entre a música em seu contexto popular e prover uma ferramenta que possa auxiliar na extração de significado para campos de pesquisa mais afastados da área de tecnologia, que muitas vezes não se valem de recursos desse tipo.
\section{Objetivos}
\label{c.objetivos}

\subsection{Objetivo Geral}
\label{c.objetivo_g}
Desenvolver um \textit{software} de extração e análise de dados que permita traçar correlações entre o cenário econômico de uma região e a tipografia de suas músicas mais populares em um dado período de tempo.

\subsection{Objetivos Específicos}
\label{c.objetivo_e}
\begin{itemize}
    \item Estudar sobre análise, mineração e redução de dados para \textit{datasets} maiores;
    \item Desenvolver o \textit{software} de redução e análise;
    \item Alimentar o produto com dados provenientes de fontes variadas;
    \item Averiguar a precisão das correlações traçadas para saber se são relevantes o bastante para ser uma ferramenta de apoio.
\end{itemize}
\chapter{Método de Pesquisa}
\label{c.metodologia}

Iniciam-se os trabalhos com a revisão bibliográfica dos assuntos relevantes à construção da ferramenta, como mineração de \textit{data sets}, \textit{machine learning} e algoritmos de análise inteligente. 

A seguir, procede-se à coleta dos conjuntos de dados, para teste, de cunho de situação econômica e de listas de streaming, e à construção da ferramenta em si, seguindo os métodos mais adequados. Passado isso, se procederão os testes, feitos com os \textit{data sets} de teste, de modo a aferir o funcionamento do programa, sendo usados os referentes ao contexto econômico e de listagem de música, sendo averiguados os resultados obtidos. A seguir, após revisão dos resultados obtidos será concluído o trabalho.
\chapter{Cronograma}
\label{c.cronograma}

Em ordem cronológica, esses serão os passos de execução:
\begin{itemize}
    \item Etapa 1: Revisão bibliográfica e pesquisa;
    \item Etapa 2: Seleção dos algoritmos e ferramentas de desenvolvimento e elaboração do software;
    \item Etapa 3: Seleção dos \textit{data sets} e teste da ferramenta;
    \item Etapa 4: Análise dos resultados;
    \item Etapa 5: Produção da monografia;
    \item Etapa 6: Entrega final;
\end{itemize}
https://pt.sharelatex.com/project/5ac1913229bb265b94508341
\begin{table}[h]
    \centering
    \begin{tabular}{|c|c|c|c|c|}
    \hline
    \textbf{Etapa} & \textbf{1º Trimestre} & \textbf{2º Trimestre} & \textbf{3º Trimestre} & \textbf{4º Trimestre} \\\hline \hline
        {Etapa 1} & {\small X} & {} & {} & {}\\\hline
        {Etapa 2} & {\small X} & {\small X} & {\small X} & {}\\\hline
        {Etapa 3} & {} & {\small X} & {\small X} & {}\\\hline
        {Etapa 4} & {} & {} & {\small X} & {\small X}\\\hline
        {Etapa 5} & {} & {} & {\small X} & {\small X}\\\hline
        {Etapa 6} & {} & {} & {} & {\small X}\\\hline
    \end{tabular}
    \caption{Fonte: Produzido pelo autor.}
    \label{tab:my_label}
\end{table}

% --------------------------------------------------------
% ELEMENTOS PÓS-TEXTUAIS
% --------------------------------------------------------

\postextual


% --------------------------------------------------------
% REFERÊNCIAS BIBLIOGRÁFICAS
% --------------------------------------------------------

\bibliography{referencias}

% --------------------------------------------------------
% GLOSSÁRIO
% --------------------------------------------------------

% Consulte o manual da classe abntex2 para orientações sobre o glossário.
%\glossary


% --------------------------------------------------------
% APÊNDICES
% --------------------------------------------------------

% Inicia os apêndices
%\begin{apendicesenv}
% Imprime uma página indicando o início dos apêndices
%\partapendices
% Criação do apêndice
%\end{apendicesenv}


% --------------------------------------------------------
% ÍNDICE REMISSIVO
% --------------------------------------------------------

\printindex


% --------------------------------------------------------
% FINAL DO DOCUMENTO
% --------------------------------------------------------

\end{document}