\section{Problema}
\label{c.problema}

Conforme discorrido, é conhecido que música gera uma resposta emocional no ouvinte e assim reflete muito sobre seu contexto, de modo que é comumente usada como objeto de estudo para melhor entender a época e o contexto socioeconômico de sua composição. O que se deseja aferir é uma relação mais indireta entre comportamentos relacionados à música, sua produção e seu consumo e o contexto social, político e econômico de um determinado local, de modo a saber se é válido, a partir da análise destes, fazer predições sobre aquele e vice-versa.
    
Com a popularização dos serviços de \textit{streaming} de música (como \textit{Spotify}, \textit{Deezer} e outros) e maior acesso a \textit{data sets} que tratam da situação econômica de uma região em um certo tempo, torna-se viável o uso de técnicas de mineração e análise de dados como ferramentas para buscar correlações entre o ambiente econômico e os tipos de peças que são compostas ou mais populares.