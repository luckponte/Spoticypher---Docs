\chapter{Introdução}
\label{c.introducao}

Música é uma parte integral da cultura humana de uma maneira quase universal, e diversos estudos apontam uma forte correlação com a psique e as emoções ~\cite{krumhansl02}, tanto como influenciadora de comportamento e sentimentos como influenciada pelos mesmos em sua construção e consumo. Desse modo, é fácil deduzir que fatores externos que também afetam essas áreas do nosso psicológico indiretamente afetam o nosso comportamento como ouvintes e produtores musicais. Desse modo, pode-s extender essa conexão comportamental e propor que tendências de composição e popularidade de peças podem ser usadas para medir e entender esses fenômenos econômicos e sociais.

Com o avanço da tecnologia da informação e comunicação, surgiu não só a necessidade de processar dados de maneira global e encontrar métodos mais inteligentes e sofisticados de processá-los para compreender seus paralelos, como também popularizam-se serviços \textit{online} para suprir, por meio de um mercado virtual, ações cotidianas, especialmente na esfera do entretenimento, entre elas: assistir vídeos, conversar ou manter contato com pessoas remotas e, naturalmente, ouvir música ~\cite{aljanaki15}.

Esse grande conjunto de dados imprecisos, e por extensão ferramentas, técnicas de processamento e manipulação, e aplicações associadas em seu estudo, denomina-se \textit{big data} ~\cite{singh15}. Até avanços maiores nas áreas de comunicação e inteligência pouco se desenvolvia sobre o assunto, dado que as estruturas comuns de armazenamento de dados (como bancos de dados) eram suficientes para a maior parte das aplicações e que, sem o uso de técnicas de aprendizagem e \textit{hardware} mais avançado e de processamento mais veloz, o processamento de tal quantidade de dados seria inviável. Entretanto, com a maior integração entre computadores, serviços e aplicações, com a  maior democratização da \textit{internet} e o advento dos dispositivos \textit{mobile}, bem como evoluções nos campos teórico e prático de IA, viu-se a oportunidade de extender esse campo, o que resultou não somente em soluções práticas para problemas do mundo real, como também em métodos de processamento e mineração de dados também aplicáveis a nível funcional para conjuntos menores de dados.

Como produtos e serviços fonográficos baseados em \textit{streaming} (isto é, a transmissão direta de dados pela \textit{internet}) e eventos de ordem econômico-social geram um grande volume de dados ~\cite{aljanaki15} que pode ser acessado para análise, pode-se usar técnicas de análise de dados provenientes do \textit{big data} para encontrar as relações propostas no primeiro parágrafo entre comportamento de consumo e produção musical e eventos que afetam a sociedade em maior escala. Esse trabalho assim se propõe a utilizar algoritmos e ferramentas de \textit{big data} e \textit{data mining} para montar um \textit{software} que processe e analise esses dados a fim de determinar se é possível traçar uma relação definida entre eventos econômicos globais e locais e tendências no mundo da música.

\chapter{Problema}
\label{c.problema}

Conforme discorrido, é conhecido que música gera uma resposta emocional no ouvinte e assim reflete muito sobre seu contexto, de modo que é comumente usada como objeto de estudo para melhor entender a época e o contexto socioeconômico de sua composição. O que se deseja aferir é uma relação mais indireta entre comportamentos relacionados à música, sua produção e seu consumo e o contexto social, político e econômico de um determinado local, de modo a saber se é válido, a partir da análise destes, fazer predições sobre aquele e vice-versa.
    
Com a popularização dos serviços de \textit{streaming} de música (como \textit{Spotify}, \textit{Deezer} e outros) e maior acesso a \textit{data sets} que tratam da situação econômica de uma região em um certo tempo, torna-se viável o uso de técnicas de mineração e análise de dados como ferramentas para buscar correlações entre o ambiente econômico e os tipos de peças que são compostas ou mais populares.
\chapter{Justificativa}
\label{c.justificativa}

Como citado, os processos de análise do contexto econômico e social ainda possuem baixa implementação de processos de automatização, o que consome tempo de pesquisa que poderia ser otimizado com ferramentas de análise de dados que auxiliassem na extração de dados e correlações significativas para o campo de estudo.

A motivação do presente trabalho é a exploração da correlação entre a música em seu contexto popular e prover uma ferramenta que possa auxiliar na extração de significado para campos de pesquisa mais afastados da área de tecnologia, que muitas vezes não se valem de recursos desse tipo.
\section{Objetivos}
\label{c.objetivos}

\subsection{Objetivo Geral}
\label{c.objetivo_g}
Desenvolver um \textit{software} de extração e análise de dados que permita traçar correlações entre o cenário econômico de uma região e a tipografia de suas músicas mais populares em um dado período de tempo.

\subsection{Objetivos Específicos}
\label{c.objetivo_e}
\begin{itemize}
    \item Estudar sobre análise, mineração e redução de dados para \textit{datasets} maiores;
    \item Desenvolver o \textit{software} de redução e análise;
    \item Alimentar o produto com dados provenientes de fontes variadas;
    \item Averiguar a precisão das correlações traçadas para saber se são relevantes o bastante para ser uma ferramenta de apoio.
\end{itemize}
