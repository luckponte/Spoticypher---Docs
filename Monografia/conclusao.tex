\chapter{Conclusão}
\label{c.conclusao}

Conforme mencionado, apesar de alguns dos resultados parecerem promissores, não pode-se com segurança confirmar uma correlação entre os dados econômicos e as características das faixas extraídos. Como pontos de consideração sobre como melhorar o desempenho em geral destacam-se, a partir do próprio relatório de erros dos experimentos, a inadequação do modelo, que muito provavelmente advém do uso de um algoritmo sub-ótimo para o problema em questão. Apesar do poder do método de SVM para classificação e da possibilidade de seu uso para regressão (como foi feito), ficou em dúvida se outros algoritmos de regressão, especialmente os especializados em tratar múltipla variáveis, não seriam mais adequados e, por consequente, retornariam resultados mais sólidos. Da mesma forma, apesar do cuidado na manipulação e extração dos dados, não se pode excluir a hipótese de viés nas amostras ou de escolha imprecisa das variáveis, tanto as econômicas quanto as pertinentes às qualidades das músicas. Também pode-se oferecer o argumento que, pesquisando o contexto de outras regiões que não apenas o Brasil poderiam se obter resultados mais conclusivos, já que o número de dados e contextos diferentes minimizariam, em tese, a influência de outros possíveis fatores nos resultados.